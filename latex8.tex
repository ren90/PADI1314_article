
%
%  $Description: Author guidelines and sample document in LaTeX 2.09$ 
%
%  $Author: ienne $
%  $Date: 1995/09/15 15:20:59 $
%  $Revision: 1.4 $
%

\documentclass[times, 10pt,twocolumn]{article} 
\usepackage{latex8}
\usepackage{times}

%\documentstyle[times,art10,twocolumn,latex8]{article}

%------------------------------------------------------------------------- 
% take the % away on next line to produce the final camera-ready version 
\pagestyle{empty}

%------------------------------------------------------------------------- 
\begin{document}

\title{PADI - DSTM}



\maketitle
\thispagestyle{empty}


%------------------------------------------------------------------------- 
\Section{Introduction}



%------------------------------------------------------------------------- 
\Section{Solution Overview}



%------------------------------------------------------------------------- 
\SubSection{Architecture}



%------------------------------------------------------------------------- 
\SubSection{Data Structures}
Our PADI-DSTM library contains multiple data structures with different goals. In the Master server are stored the list of transactions unique identifiers that started and ended, in order to have a log of the transactions that have been completed, and if one transaction id only exists in the list of started transactions, this means that the transaction has been aborted. The servers continually send "I'm alive" messages to the Master which updates a list that contains the active servers, and when the client connects with the Master by calling the methods AccessPadInt or CreatePadInt, the Master only returns the active servers in the object that contains the PadInt. This PadInt object is the main one in our architecture, it is stored in the servers, and each PadInt exists in three servers, in order to tolerate faults. Each of the three objects is always updated to the last version, because when the changes to the object are committed, the new values are propagated to the other servers in which the object is stored. The PadInt object contains different sub-objects, its UID (to make each PadInt unique), its value, an array of the servers in which it is stored (we decided that each object know its own location, and the client decides which one to bind to, when he receives the object through the API). The PadInt object also contains a timestamp that it is used to control the accesses to the variable in the memory, in this case if the client timestamp is lower than the timestamp of the object it means that the client has an old version of the object and the transaction is aborted.

%------------------------------------------------------------------------- 
\SubSection{Responsibilities}

%------------------------------------------------------------------------ 
\SubSection{Concurrency Control}

%------------------------------------------------------------------------- 
\SubSection{Transactional Algorithms}

%------------------------------------------------------------------------- 
\SubSection{Distributed Protocols}

%------------------------------------------------------------------------- 
\SubSection{Deadlock Detection and Abort Recovery}
Bearing in mind that we're using timestamps to prevent concurrent access to shared objects, the greatest advantage when using this type of concurrency control is that there are no deadlock situations, since the access to shared objects aren't locked by another data type, like the locks used in database transactions or the OS's mutexes. Even though this concurrency control method avoids deadlock through its implementation, there are some situations where the occurring transactions abort without a conflict of two (or more) of those transactions accessing the same shared object. This sort of thing happens when, for example, a transaction T1 has a timestamp value of 100 and another transaction T2 has a timestamp value of 200. If T2 tries to read an object (without altering its value) before T1, though there are no read-write/write-write/write-read conflicts, both T1 and T2 must abort because the access to the object wasn't sequential (since T1 has a lower timestamp value than T2).

\Section{Conclusions}

%------------------------------------------------------------------------- 

\nocite{ex1,ex2}
\bibliographystyle{latex8}
\bibliography{latex8}

\end{document}

