
%
%  $Description: Author guidelines and sample document in LaTeX 2.09$ 
%
%  $Author: ienne $
%  $Date: 1995/09/15 15:20:59 $
%  $Revision: 1.4 $
%

\documentclass[times, 10pt,twocolumn]{article} 
\usepackage{latex8}
\usepackage{times}

%\documentstyle[times,art10,twocolumn,latex8]{article}

%------------------------------------------------------------------------- 
% take the % away on next line to produce the final camera-ready version 
\pagestyle{empty}

%------------------------------------------------------------------------- 
\begin{document}

\title{PADI - DSTM}



\maketitle
\thispagestyle{empty}


%------------------------------------------------------------------------- 
\Section{Introduction}



%------------------------------------------------------------------------- 
\Section{Solution Overview}



%------------------------------------------------------------------------- 
\SubSection{Architecture}



%------------------------------------------------------------------------- 
\SubSection{Data Structures}
Our PADI-DSTM library vontains multiple data structures with different goals. In the Master server are stored the list of transactions unique identifiers that started and ended, in order to have a log of the transactions that have been completed, and if one transaction id only exits in the list of started transactions, this means that the transaction have been aborted. The servers continually sends "I'm alive" messages to the Master which updates a list that contains the active servers, and when the client connects the master with an AccessPadInt and CreatePadInt, the Master only returns the active servers in the object that contains the PadInt. This PadInt object is the main object in our architecture, he is stored in the servers, and each PadInt exists in three servers, in order to tolerate faults. Each of the three objects is always in the last version, because when the changes to the object are commited, the new values are propagated to the other servers in which the object is stored. The PadInt object contains different sub-objects, its UID, to make each PadInt unique, its value, an array of the servers in which it is stored, we decided that each object know it's own location, and the client decides which one to bind, when he receives the object from the API. The PadInt object also contains a timestamp that it is used to controlling acesses to the variable in the memory, in this case if the client timestamp is lower than the timestamp of the object it means that the client has an old version of the object and the transaction is aborted.

%------------------------------------------------------------------------- 
\SubSection{Responsibilities}

%------------------------------------------------------------------------ 
\SubSection{Concurrency Control}

%------------------------------------------------------------------------- 
\SubSection{Transactional Algorithms}

%------------------------------------------------------------------------- 
\SubSection{Distributed Protocols}


%------------------------------------------------------------------------- 
\SubSection{Deadlock Detection and Abort Recovery}

\Section{Conclusions}

%------------------------------------------------------------------------- 

\nocite{ex1,ex2}
\bibliographystyle{latex8}
\bibliography{latex8}

\end{document}

